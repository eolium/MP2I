\documentclass{article}

\usepackage{hyperref}

\hypersetup{
    hidelinks,
    colorlinks=true,
    linkcolor=blue,
    filecolor=magenta,      
    urlcolor=cyan,
    pdftitle={Table des matières},
}

\input{lib.tex}

\geometry{hmargin=2.2cm, vmargin=3.7cm}

\pagestyle{fancy}
\fancyhead[L]{EC2 - Régime transitoire des circuits linéaires}
\fancyhead[R]{Yoan de CORNULIER}

\title{Cinétique : Suite du cours}
\date{}

\begin{document}

\subsubsection{Système de coordonnées cylindriques}


Pour décrire certains mouvements, on préférera un autre système de coordonnées, basé sur les coordonnées polaires. On pose :

\begin{align*}\begin{split}
\begin{cases}
    r = OH\in\R\\\\
    \theta = \Bigl(\overrightarrow{u_x},\overrightarrow{OH}\Bigl)\in[0;2\pi[ \text{ou} ]-\pi;\pi]\\\\
    z\in\R
\end{cases}
\end{split}\end{align*}

On se demande alors quelle est la base du système (BOND). On pose :

\begin{align*}\begin{split}
\overrightarrow{u_r}=\frac{\overrightarrow{OH}}{OH}
\end{split}\end{align*}

\bigskip
\bigskip

\subsubsection*{Lien entre coordonnées cartésiennes et polaires}

On a :

\begin{align*}\begin{split}
\begin{cases}
    r=\sqrt{x^2+y^2}\\\\
    \theta=\begin{cases}
        \arctan\frac{y}{x},x>0\\
        \arctan\frac{y}{x}+\pi,x<0\\
    \end{cases}
\end{cases}
\end{split}\end{align*}

On peut alors poser :

\begin{align*}\begin{split}
\overrightarrow{u_r}=\cos\theta\overrightarrow{u_x}+\sin\theta\overrightarrow{u_y}
\overrightarrow{u_\theta}=-\sin\theta\overrightarrow{u_x}+\cos\theta\overrightarrow{u_y}
\end{split}\end{align*}


\end{document}