\documentclass{article}

\usepackage{hyperref}

\hypersetup{
    hidelinks,
    colorlinks=true,
    linkcolor=blue,
    filecolor=magenta,      
    urlcolor=cyan,
    pdftitle={Table des matières},
}

\input{lib.tex}

\geometry{hmargin=2.2cm, vmargin=3.7cm}

\pagestyle{fancy}
\fancyhead[L]{EC2 - Régime transitoire des circuits linéaires}
\fancyhead[R]{Yoan de CORNULIER}

\title{Cinétique : Suite du cours}
\date{}

\begin{document}

\subsubsection{Système de coordonnées cylindriques}


Pour décrire certains mouvements, on préférera un autre système de coordonnées, basé sur les coordonnées polaires. On pose :

\begin{align*}\begin{split}
\begin{cases}
    r = OH\in\R\\\\
    \theta = \Bigl(\overrightarrow{u_x},\overrightarrow{OH}\Bigl)\in[0;2\pi[ \text{ou} ]-\pi;\pi]\\\\
    z\in\R
\end{cases}
\end{split}\end{align*}

On se demande alors quelle est la base du système (BOND). On pose :

\begin{align*}\begin{split}
\overrightarrow{u_r}=\frac{\overrightarrow{OH}}{OH}
\end{split}\end{align*}

\bigskip
\bigskip

\subsubsection*{Lien entre coordonnées cartésiennes et polaires}

On a :

\begin{align*}\begin{split}
\begin{cases}
    r=\sqrt{x^2+y^2}\\\\
    \theta=\begin{cases}
        \arctan\frac{y}{x},x>0\\
        \arctan\frac{y}{x}+\pi,x<0\\
    \end{cases}
\end{cases}
\end{split}\end{align*}

On peut alors poser :

\begin{align*}\begin{split}
\overrightarrow{u_r}=\cos\theta\overrightarrow{u_x}+\sin\theta\overrightarrow{u_y}
\overrightarrow{u_\theta}=-\sin\theta\overrightarrow{u_x}+\cos\theta\overrightarrow{u_y}
\end{split}\end{align*}


\subsubsection*{Concernant $d\overrightarrow{u_r}\dt$ :}

On pose :

\begin{align*}\begin{split}
\overrightarrow{u_r}.\overrightarrow{u_r}=1\\
2\frac{d\overrightarrow{u_r}}{dt}.\overrightarrow{u_r}=0\\
\longrightarrow \frac{d\overrightarrow{u_r}}{dt}\perp\overrightarrow{u_r}
\end{split}\end{align*}

On en déduit :

\begin{align*}\begin{split}
d\overrightarrow{u_r}=k\times\overrightarrow{u_\theta}\\
d\overrightarrow{u_r}=d\theta\overrightarrow{u_r}\\
\end{split}\end{align*}

D'où :

\begin{align*}\begin{split}
\boxed{\frac{d\overrightarrow{u_r}}{dt}=\dot{\theta}\overrightarrow{u_\theta}}
\end{split}\end{align*}

De même :

\begin{align*}\begin{split}
\frac{d\overrightarrow{u\theta}}{dt}=-\dot{\theta}\overrightarrow{u_r}
\end{split}\end{align*}

On peut donc en conclure :

\begin{align*}\begin{split}
\boxed{\overrightarrow{v}=\dot{r}\overrightarrow{u_r}+r\dot{\theta}\overrightarrow{u_\theta}+\dot{z}\overrightarrow{u_z}}
\end{split}\end{align*}


\subsubsection*{Expression de l'accélération :}

On pose :

\begin{align*}\begin{split}
v=\sqrt{\dot{r}^2+r^2\dot{\theta}^2+\dot{z}^2}\\
\boxed{d\overrightarrow{l}=dr\overrightarrow{u_r}+rd\theta\overrightarrow{u_\theta}+dz\overrightarrow{u_z}}
\end{split}\end{align*}

On en déduit alors sa dérivée :

\begin{align*}\begin{split}
\overrightarrow{a}=\frac{d\overrightarrow{v}}{dt}=\frac{d}{dt}\Bigl(\dot{r}\overrightarrow{u_r}+r\dot\theta\overrightarrow{u_\theta}+\dot{z}\overrightarrow{u_z}\Bigl)\\
\overrightarrow{a}=\ddot{r}\overrightarrow{u_r}+\dot{r}\frac{d\overrightarrow{u_r}}{dt}+(\dot{r}\dot{\theta}+r\ddot{theta})\overrightarrow{u_\theta}+r\dot{\theta}\frac{d\overrightarrow{u_\theta}}{dt}+\ddot{z}\overrightarrow{u_z}\\\\
\boxed{\overrightarrow{a}=(\ddot{r}-r\dot{\theta}^2)\overrightarrow{u_r}+(r\ddot{\theta}+2\dot{r}\dot{\theta})\overrightarrow{u_\theta}+\ddot{z}\overrightarrow{u_z}}
\end{split}\end{align*}

On simplifie le vecteur accélération (\remind{à apprendre par coeur !}) :

\begin{align*}\begin{split}
\left(
\begin{array}{c}
    \ddot{r}-r\dot{\theta}^2\\
    r\ddot{\theta}+2\dot{r}\ddot{\theta}\\
    \ddot{z}
\end{array}
\right)
\end{split}\end{align*}

\subsection{Système de coordonnées et base sphérique :}

Pour se repérer sur Terre, on utilise le système de latitude/longitude, comme suit :

\begin{itemize}
    \item $\varphi$ : Longitude, défini sur $]-\pi;\pi[$.
    \item $\lambda$ : Latitude, défini sur $]-\frac{\pi}{2};\frac{\pi}{2}[$
\end{itemize}

Le système de coordonnées sphérique se pose alors par :

\begin{itemize}
    \item $r=OM\in\R^+$ : Rayon
    \item $\theta=\hat(\overrightarrow{u_z},\overrightarrow{OM})\in [0;\pi ]$ : Colatitude
    \item $\varphi=\hat(\overrightarrow{u_x},\overrightarrow{OH})\in [0;2\pi [$ ou $\in ]-\pi;\pi ]$ : Longitude
\end{itemize}

\remind{Attention, ici l'angle $\varphi$ correspond à l'angle $\theta$ du système de coordonnées cylindriques}


\underline{Remarque : on peut passer du système sphérique au système cartésien ou cylindrique}

\begin{itemize}
    \item 
        \begin{align*}\begin{split}
        r = \sqrt{x^2+y^2+z^2}\\
        \cos\theta=\frac{z}{\sqrt{x^2+y^2}}\\
        \tan\varphi=\frac{y}{x}
        \end{split}\end{align*}

    \item
        \begin{align*}\begin{split}
            z=r\cos\theta\\
            y=r\sin\theta\sin\varphi\\
            x=r\sin\theta\cos\varphi
        \end{split}\end{align*}
\end{itemize}

\underline{Vecteur position :}

$$
\boxed{\overrightarrow{OM}=r\overrightarrow{u_r}}
$$

$$
\boxed{\overrightarrow{OM}=\left(\begin{array}{c}r\\0\\0\end{array}\right)}
$$

\underline{Vecteur déplacement élémentaire :}

$$
d\overrightarrow{l}=dr\overrightarrow{u_r}+rd\theta\overrightarrow{u_\theta}+r\sin\theta d\varphi\overrightarrow{u_\varphi}
$$

On en déduit l'expression du vecteur vitesse :

$$
\boxed{\overrightarrow{v}=\dot{r}\overrightarrow{u_r}+r\dot{\theta}\overrightarrow{u_\theta}+r\sin\theta\dot{\varphi}\overrightarrow{u_\varphi}}
$$

On peut alors l'exprimer plus simplement :

$$\left(\begin{array}{c}\dot{r}\\r\dot{\theta}\\r\sin\theta\dot{\varphi}\end{array}\right)_\text{sph}$$

\section{Base de Frénet : (2D)}

\subsubsection*{Abscisse curviligne}

$s(t)$ est la distance parcourue le long de $\Ts$ le mouvement. On en déduit la base de Frénet :

\begin{itemize}
    \item $\overrightarrow{u_T}$ vecteur tangeant à $\Ts$ dans le sens du mouvements
    \item $\overrightarrow{u_n}$ vecteur unitaire normal à $\Ts$ dirigié vers la concavité de $\Ts$.
\end{itemize}

La base est donc mobile, locale et intrinsèque.

\subsubsection*{Vecteur déplacement élémentaire}

$$ds(t) = s(t+dt)-s(t)$$

$$\boxed{d\overrightarrow{l}=ds\overrightarrow{u_T}}$$

\underline{Vecteur vitesse}

$$\boxed{\overrightarrow{v}=\dot{s}\overrightarrow{u_T}}\Longrightarrow\boxed{v=\frac{ds}{dt}}
\Longrightarrow
\boxed{\overrightarrow{v}=\left(\begin{array}{c}\dot{s}\\0\end{array}\right)}$$

Vecteur accélération :

$$
\overrightarrow{a}=\frac{d\overrightarrow{v}}{dt}
\Longrightarrow \overrightarrow{a}=\ddot{s}\overrightarrow{u_T}+\dot{s}\frac{d\overrightarrow{u_T}}{dt}
$$

Comme $||\overrightarrow{u_T}||=1$ (constante) en $a$, on a :

$$\frac{d\overrightarrow{u_T}}{dt} = k\times\overrightarrow{u_N}$$



\end{document}