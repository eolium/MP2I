\documentclass{article}

\usepackage{hyperref}

\hypersetup{
    hidelinks,
    colorlinks=true,
    linkcolor=blue,
    filecolor=magenta,      
    urlcolor=cyan,
    pdftitle={Table des matières},
}

\input{lib.tex}

\geometry{hmargin=2.2cm, vmargin=3.7cm}

\pagestyle{fancy}
\fancyhead[L]{EC2 - Régime transitoire des circuits linéaires}
\fancyhead[R]{Yoan de CORNULIER}

\title{EC2 : Régime transitoire des circuits linéaires}
\date{}

\begin{document}
\maketitle
\tableofcontents

\fancyhead[L]{M1 - Éléments de cinématique}
\addcontentsline{toc}{section}{Introduction}
\section*{Introduction}

\bigskip
\bigskip


\section[4]{Différentes bases et systèmes de coordonnées}

\subsection{Base et coordonnées cartésiennes}

\remind{Figure 1}

BOND : Base OrthoNormée Directe

On pose :

\begin{align*}\begin{split}
||\overrightarrow{u_x}||=||\overrightarrow{u_y}||=||\overrightarrow{u_z}||=1
\end{split}\end{align*}

afin que les distances unitaires valent 1, et que les produits scalaires des vecteurs unitaires valent 0.




\end{document}