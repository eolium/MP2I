\documentclass{article}

\usepackage{hyperref}

\hypersetup{
    hidelinks,
    colorlinks=true,
    linkcolor=blue,
    filecolor=magenta,      
    urlcolor=cyan,
    pdftitle={Table des matières},
}

\input{lib.tex}

\geometry{hmargin=2.2cm, vmargin=3.7cm}

\pagestyle{fancy}
\fancyhead[L]{EC2 - Régime transitoire des circuits linéaires}
\fancyhead[R]{Yoan de CORNULIER}

\title{Mécanique : Partie 3}
\date{}

\begin{document}

\section{Étude des petites oscillations autour d'une position stable}

\underline{Idée : } Montrer qu'autour d'une position d'équation stable, le mouvement se ramène à une première approximation à celui d'un oscillateur harmonique.

Il faut "paraboliser" le puit de potentiel.

\subsection*{Approximation d'une fonction par un polynôme de degré 2}

\begin{align*}\begin{split}
f'(x)\approx f'(x_0) + f''(x_0) \times (x-x_0)\\\\
f(x)\approx f'(x_0)(x-x_0)+\frac{f''(x_0)(x-x_0)^2}{2}+C\\
\boxed{f(x)\approx f(x_0)+f'(x_0)(x-x_0)+\frac{f''(x_0)}{2}(x-x_0)^2}
\end{split}\end{align*}

Soit $r=r_{éq}$ une position d'équilibre. On a :

\begin{align*}\begin{split}
\frac{d\Es_p}{dr}(r_{éq})=0\\
\Longrightarrow \Es_p\approx\Es_p(r_{éq})+\frac{d\Es_p}{dr}(r_{éq})(r-r_{éq})
\end{split}\end{align*}

Finalement, on a :

\begin{align*}\begin{split}
\Es_p(r)\approx\Es_p(r_{éq})+\frac{1}{2}\frac{d^2\Es_p}{dr^2}(r_{éq})(r-r_{éq})
\end{split}\end{align*}

Or :

\begin{align*}\begin{split}
\Es_m = \Es_c + \Es_p = C\\\\
\frac{1}{2}m\dot{r}^2+\Es_p(r_{éq})+\frac{1}{2}\frac{d^2\Es_p}{dr^2}(r_{éq})(r-r_{éq})^2 = C
\end{split}\end{align*}

Pour simplifier, notons $\epsilon = r-r_0$, alors $\dot{\epsilon}=\dot{r}$.

\begin{align*}\begin{split}
\frac{1}{2}m\dot{\epsilon}^2+\frac{1}{2}\frac{d^2\Es_p}{dr^2}(r_{éq})\epsilon^2=C
\end{split}\end{align*}

On dérive par rapport au temps :

\begin{align*}\begin{split}
m\dot{\epsilon}\ddot{\epsilon}+\frac{d^2\Es_p}{dr^2}(r_{éq})\dot{\epsilon}\epsilon = 0
\end{split}\end{align*}


Finalement, on peut en conclure que :

\begin{align*}\begin{split}
\boxed{
\ddot{\epsilon}+\omega^2\epsilon=0,\\
\omega^2=\frac{\frac{d^2}{}}{}
}
\end{split}\end{align*}


\end{document}