\documentclass[10pt,dvipsnames,svgnames]{article}

\setlength{\parindent}{0pt}

\title{Titre inutile}

\usepackage{geometry}
\usepackage{tikz}
\usepackage[european, straightvoltages, cute inductors]{circuitikz}

\geometry{hmargin=2.2cm,vmargin=3.7cm}

\usepackage[utf8]{inputenc}

\usepackage{amsfonts}
\usepackage{amsmath,amssymb}

\usepackage{graphicx}


\usepackage{tabularx}

\newcommand{\vsa}{\vspace{1em}}
\newcommand{\vsb}{\vspace{0.5em}}


\usepackage{fancyhdr}
\usepackage{fancybox}
\pagestyle{fancy}
\fancyhead[L]{Correction TD - Traitement du signal}
\fancyhead[R]{Ewen Passelande}

\renewcommand*{\thesubsection}{\arabic{subsection}.}

\begin{document}

\section*{Exercice 4 - Comportement d'un circuit}

\newcolumntype{M}[1]{>{\raggedright}m{#1}}
\begin{tabular}{M{12cm} M{10cm}}
  Soit le circuit ci-à-côté. Le condensateur est initialement déchargé et on ferme l'interrupteur à $ t=0 $. Les différentes quantités $ i(t) $, $ i_1 (t) $, $ i_2 (t) $ et $ q(t) $ (charge du condensateur) vérifient une équation différentielle linéaire du deuxième ordre à coefficient constants (que l'on ne cherchera pas à établir) dont la solution homogène est pseudo-oscillante.
  &
    \begin{circuitikz}
      \draw (0,-3)
      -- (0.2,-3)
      to [closing switch, l_=$K$] (2,-3)
      to [vsource, v_=$E$] (4,-3)
      -- (4,-1);

      \draw (0,-3)
      -- (0,-1)
      -- (0.3,-1)
      -- (0.3,-0.5)
      to [L, l=$L$, i^<=$i_1$] (2.1,-0.5)
      -- (2.1,-1)
      to [R, v=$Ri$, l=$R$] (4,-1);

      \draw (0.3,-1)
      -- (0.3,-1.5)
      to [C, l_=$C$, i^<=$i_2$, v^>=$u$] (2.1, -1.5)
      -- (2.1, -1);
      
      \draw (4,-3)
      to [open, i=$i$] (4,-2);
    \end{circuitikz}
\end{tabular}

\subsection{Déterminer en $ t = 0^+ $ les valeurs des différentes intensités, de la charge ainsi que de $ \dfrac{di}{dt} $.}

En $ t = 0^+ $, d'après la continuité du courant aux bornes de la bobine, \fbox{$ i_1(0^+) = 0 $}.

De plus, la charge du condensateur est aussi continue, donc \fbox{$ q(0^+) = 0 $}.


On trouve donc le circuit équivalent suivant :

\begin{tabular}{M{5cm} M{5cm}}
  \begin{circuitikz}
    \draw (0,0)
    to [vsource, v_=$E$] (3,0)
    -- (3,2)
    to [R, v<=$Ri$] (1,2)
    -- (0,2)
    -- (0,0);
    
    \draw (3,0)
    to [open, i=$i$] (3,2);
    
    \draw (1,2)
    to [open, i_>=$ i_2 $] (0.5,2);
    
    \draw (1,1.8)
    to [open, v^<= {$ u = 0 $}] (0,1.8);
  \end{circuitikz}  
  
  & \[
    \begin{array}{c c @{ \hspace{2em} \Longrightarrow \hspace{2em} } c}
      
      \text{Ainsi :} &
                       \left \{
                       \begin{array}{c @{=} c}
                         i(0^+) & i_2(0^+) \\
                         E & R i(0^+)
                       \end{array}
                             \right.
                           &
                             \fbox{$i(0^+) = i_2(0^+) = \dfrac{E}{R}$}
    \end{array} \raggedright
                             \]\\
\end{tabular}

\begin{tabular}{c l} 
  De plus, d'après la loi des mailles : \hspace {1em} $ E = u + Ri $
  & $ \Longrightarrow E = \dfrac{q}{C} + Ri $ \\
  & $ \Longrightarrow 0 = \dfrac{1}{C} \dfrac{dq}{dt} + R \dfrac{di}{dt} \hspace{1em} \text{(en dérivant)} $ \\
  & $ \Longrightarrow \dfrac{i_2}{C} + R \dfrac{di}{dt} = 0 $ \\
  & $ \Longrightarrow \fbox{$\dfrac{di}{dt}(0^+) = - \dfrac{i_2(0^+)}{RC} = 0$} $
\end{tabular}\\

\newpage
\subsection{Faire de même après un temps très long, c'est-à-dire après le régime transitoire.}

Après que le régime permanent ait été établi, la bobine se comporte comme un fil et le condensateur comme un interrupteur ouvert, on a donc \fbox{$i_2(+\infty) = 0$} et on obtient le circuit suivant :

\begin{tabular}{M{5cm} M{5cm}}
  \begin{circuitikz}
    \draw (0,0)
    to [vsource, v_=$E$] (3,0)
    -- (3,2)
    to [R, v<=$Ri$] (1,2)
    -- (0,2)
    -- (0,0);
    
    \draw (3,0)
    to [open, i=$i$] (3,2);
    
    \draw (1,2)
    to [open, i_>=$ i_1 $] (0.5,2);
    
    \draw (1,1.8)
    to [open, v^<= {$ u = 0 $}] (0,1.8);
  \end{circuitikz}  
  
  & \[
    \begin{array}{c c @{ \hspace{2em} \Longrightarrow \hspace{2em} } c}
      
      \text{Ainsi :} &
                       \left \{
                       \begin{array}{c @{=} c}
                         i(+\infty) & i_1(+\infty) \\
                         E & R i(+\infty)
                       \end{array}
                             \right.
                           &
                             \fbox{$i(+\infty) = i_1(+\infty) = \dfrac{E}{R}$} \\
      \text{Et :} & u = 0 & \fbox{$ q(+\infty) = Cu(+\infty) = 0$}
    \end{array} \raggedright
                             \]\\
\end{tabular}

Et enfin, comme le régime permanent est établi, les grandeurs ne varient plus, donc \fbox{$ \dfrac{di}{dt}(+\infty) = 0$}

\subsection{En déduire l'allure de l'évolution temporelle de l'intensité $i(t)$.}

On obtient finalement d'après les deux premières questions :
\renewcommand{\arraystretch}{2.3}
\[
\hspace{1em}
\begin{array}{|c|c|c|c|c|c|}
  \hline
  & i &  i_1  &  i_2  &  q  &  \dfrac{di}{dt}   \\
  \hline
  t = 0^+  &  \dfrac{E}{R}  &  0  &  \dfrac{E}{R}  &  0  &  0  \\
  \hline
  t = +\infty & \dfrac{E}{R} & \dfrac{E}{R} & 0 & 0 & 0 \\
  \hline
\end{array}
\]

$i(0^+) = \dfrac{E}{R}$ et $i(t)$ tend vers $\dfrac{E}{R}$, de plus la dérivée de $i$ est négative en $t=0^+$. Comme $i$ est pseudo-oscillante d'après l'énnoncé, on obtient l'allure suivante :

\begin{center}
  \begin{tikzpicture}[domain=0:7]
    \draw[->] (-1,0) -- (7,0) node[right] {$t$};
    \draw[->] (0,-1) -- (0,3) node[above] {$i(t)$};
    \draw[dotted] (0,1.5) -- (7,1.5) node[above] {$ $};
    
    \foreach \y in {1.5} {
      \draw (0.1cm,\y) -- (-0.1cm,\y) node[left] {$\dfrac{E}{R}$};
    }
    
    \begin{scope}
      \clip (0,-1) rectangle (10,4);
      \draw[color=red,samples=100] plot ({\x},{1.5-exp(-(\x/1.4))*sin(150*\x)});
      
    \end{scope}
  \end{tikzpicture}
\end{center}

\end{document}