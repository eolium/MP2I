\documentclass{article}

\input{lib.tex}

\title{Exercices Du TD de Mathématiques}
\author{Arnaud, Hugo, Ki Mi, Raphaël, Yoan}

\begin{document}
\maketitle

\bigskip
\bigskip

\underline{Exercice 12 :}

\bigskip
\bigskip

On pose :

\begin{itemize}
    \item $f:[0;+\infty[\longrightarrow+\infty$ continue sur $\R$
    \item $|f|$ tend vers $+\infty$.
\end{itemize}


\bigskip
\bigskip
\bigskip
\bigskip


Montrons d'abord que $f$ diverge, puis montrons que $f$ que $f$ tend vers $+\infty$ ou $-\infty$.

\begin{itemize}
    \item Supposons par l'absurde que $f$ est bornée. Alors :
    
    \begin{align*}\begin{split}
        \exists(A, B)\in\R^2 / \forall x\in\R, A\leq f(x)\leq B\\
        \Longrightarrow  |f(x)| \leq |A|+|B|
    \end{split}\end{align*}
        
    Ainsi, $f$ est bornée, ce qui est absurde, puisque $f$ tend vers $+\infty$.

    On en conclut que $f$ diverge.


    \item On pose :
    
    \begin{align*}\begin{split}
        \lim_{x\rightarrow+\infty}|f(x)|=+\infty\\
        \Longrightarrow \forall M\geq0,\exists A\ge0/\forall x\ge A,|f(x)|>M\\
        \Longrightarrow f(x)<-M ~~~\text{ou}~~~ f(x)>M
    \end{split}\end{align*}

    Supposons par l'absurde que :

    \begin{align*}\begin{split}
        \exists x>A/f(x)<-M<0\\
        \exists y>A/f(y)>M>0
    \end{split}\end{align*}

    or $f$ est continue sur $R$, ainsi par théorème des valeurs intermédiaires, on en déduit :

    \begin{align*}\begin{split}
        \exists z>A/f(z)=0\\
        \Longrightarrow \boxed{|f(z)|=0}
    \end{split}\end{align*}
    
    Or :


    \begin{align*}\begin{split}
        \forall x>A, \boxed{|f(x)|>M>0}
    \end{split}\end{align*}

    On arrive à une absurdité, on en conclut donc que $f$ tend vers $\pm\infty$.
\end{itemize}

Ainsi, on a montrée que $f$ tend vers $\pm\infty$.

\bigskip
\bigskip
\bigskip

\underline{Exercice 17 :}

On pose :

\begin{itemize}
    \item $f:\R\longrightarrow\R$ une fonction $k$-lipschitzienne ($k\in[0;1[$), telle que $f(0)=0$.
    \item $a\in\R$.
    \item $(u_n)\in\R^\N$ telle que $u_0=a$, $\forall n\in\N, u_{n+1}=f(u_n)$.
\end{itemize}

Montrons que $(u_n)$ converge vers 0.

\bigskip
\bigskip
\bigskip
\bigskip

$f$ est lipschitzienne, donc :

$$\forall(x, y)\in\R^2, |f(x)-f(y)|\leq k|x-y|$$

En particulier, $\forall n\in\N$ :

\begin{align*}\begin{split}
|f(x)|\leq k|x|\\
\Longrightarrow |u_{n+1}|\leq k|u_n|\\
\end{split}\end{align*}

\bigskip

Montrons alors par récurrence $P(n\in\N):"|u_n|\leq k^n |a|"$.

\begin{itemize}
    \item Posons $n=0$. Alors :
    $$|u_0|=|a| \leq |a|$$

    \item Soit $n\in\N/P(n)$ est vraie. On a alors :
    \begin{align*}\begin{split}
        |u_n|\leq k^n|a|\\
        \Longrightarrow |u_{n+1}|\leq k|u_n| \leq k^{n+1} |a|\\
        \Longrightarrow \boxed{|u_{n+1}|\leq k^{n+1}|a|}
    \end{split}\end{align*}

    \item En conclusion, on a montré que :
    
    $$\forall n\in\N, |u_n|\leq k^n|a|$$
\end{itemize}

Or la suite $(k^n|a|)$ est géométrique de raison $k\in[0;1[$, donc elle converge vers 0. Ainsi, $|u_n|$ est majorée par une suite qui converge vers 0 et par 0 (par propriétés de la valeur absolue), par théorème de l'encadrement, elle converge donc vers 0.

\bigskip

On a donc montré que $(u_n)$ converge vers 0.

\end{document}