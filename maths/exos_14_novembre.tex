\documentclass{article}

\input{lib.tex}

\title{Exercices Du TD de Mathématiques}
\author{Adam, Lilian, Raphaël, Yoan}

\begin{document}
\maketitle

\underline{Exercice 10 :}

Soit $f$ une fonction définie et dérivable sur $\R$. On pose :

\begin{align}\begin{split}
\forall x\in\R f'(x)&=f(-x)\\\\
\Longrightarrow \forall x\in\R f''(x)&=f'(-x)=-f'(x)\\\\
\Longrightarrow \forall x\in\R f''(x)+f'(x)&=0
\end{split}\end{align}

Déterminons l'ensemble des solutions de cette équation différentielle. On pose pour cela l'équation caractéristique dans $\C$ :

\begin{align*}\begin{split}
r^2+1=0
\end{split}\end{align*}

On a alors $r_1=i$ et $r_2=-i$. On peut alors en déduire l'ensemble des solutions :

\begin{align*}\begin{split}
\boxed{\Ss_1=\Bigl\{\lambda e^{it}+\mu e^{-it}, (\lambda,\mu)\in\K \Bigl\}}
\end{split}\end{align*}

\bigskip
\bigskip
\bigskip

Soit $f:\begin{cases}\R\longrightarrow\K \\ t\longrightarrow \lambda e^{it} + \mu e^{-it}\end{cases}$ une fonction dérivable sur $\R$ et :

\begin{align*}\begin{split}
\forall x\in\R, f'(x)=\lambda ie^{it}-\mu ie^{-it}\\
\end{split}\end{align*}

On pose alors :

\begin{align*}\begin{split}
\forall x\in\R, f'(x)&=f(-x)\\\\
\Longleftrightarrow \lambda i e^{it}-\mu ie^{-it} &= \lambda e^{it}+\mu e^{-it}\\\\
\Longleftrightarrow \lambda e^{it} - i\lambda e^{it} +\mu e^{-it} + i\mu e^{-it}&=0\\\\
\Longleftrightarrow \lambda - i\lambda +\mu e^{-2it}(1+i)&=0\\\\
\Longleftrightarrow \lambda-i\lambda +\mu(\cos(-2t)+i\sin(-2t))(1+i)&=0\\\\
\Longleftrightarrow \lambda-i\lambda +\mu(\cos(2t)-i\sin(2t))(1+i)&=0\\\\
\Longleftrightarrow \lambda-i\lambda +\mu\cos(2t)-i\mu\sin(2t)+i\mu\cos(2t)+\mu\sin(2t)&=0\\\\
\Longleftrightarrow \begin{cases}
\lambda+\mu(\cos(2t)+\sin(2t))&=0\\
-\lambda+\mu(cos(2t)-sin(2t))&=0
\end{cases}\\\\
\end{split}\end{align*}

Posons alors $x=0$ :

\begin{align*}\begin{split}
&\Longrightarrow \begin{cases}
    \lambda + \mu(1+0)=0\\
    -\lambda + \mu(1-0)=0
\end{cases}\\\\
&\Longrightarrow \begin{cases}
    \lambda = -\mu\\
    \lambda = \mu
\end{cases}\\\\
&\Longrightarrow \begin{cases}
    \lambda = 0\\
    \mu = 0
\end{cases}
\end{split}\end{align*}

Finalement, on peut en conclure que l'ensemble des fonctions qui vérifient $\forall x\in\R, f'(x)=f(-x)$ est :

\begin{align*}\begin{split}
\Ss = \boxed{\Bigl\{x\longrightarrow 0\Bigl\}}
\end{split}\end{align*}

\bigskip
\bigskip
\bigskip

\underline{Exercice 7 :}

\bigskip
\bigskip
\bigskip

On pose :

\begin{align*}\begin{split}
z=\frac{1}{y}\Longrightarrow z'=\frac{-y}{y^2}
\end{split}\end{align*}

Ainsi, $\forall x\in]0;+\infty[$ :

\begin{align*}\begin{split}
xy'+3y&=x^2y^2\\
\Longleftrightarrow -\frac{xy'}{y^2}-3\frac{1}{y}&=x^2\\
\Longleftrightarrow xz'-3z&=-x^2\\
\Longleftrightarrow z'-\frac{3}{x}z&=-x
\end{split}\end{align*}

\begin{align*}\begin{split}
\Ss_0=\Bigl\{\lambda e^{3\ln(x)}, \lambda\in\K\Bigl\}=\Bigl\{\lambda x^3, \lambda\in\K\Bigl\}
\end{split}\end{align*}

\bigskip
\bigskip

Soit $y:\begin{cases}\R\longrightarrow\K\\x\longrightarrow\lambda(x)x^3\end{cases}$. On pose :

\begin{align*}\begin{split}
y\in\Ss &\Longleftrightarrow \lambda(x)'x^3=x\\
&\Longleftrightarrow \lambda(x)'=-\frac{1}{x^2}\\
\end{split}\end{align*}

On pose alors :

\begin{align*}\begin{split}
\lambda(x)=\frac{1}{x}
\end{split}\end{align*}

On peut donc en déduire l'ensemble des solutions :

\begin{align*}\begin{split}
\Ss_1=\Bigl\{x\longrightarrow\lambda x^3+\frac{1}{x}, \lambda\in\K\Bigl\}
\end{split}\end{align*}

Or on a $y\frac{1}{z}$. On peut donc en déduire l'ensemble des solutions possibles :

\begin{align*}\begin{split}
\Ss_2=\Bigl\{x\longrightarrow\frac{1}{\lambda x^3+x^2},\lambda\in\K\Bigl\}
\end{split}\end{align*}

\bigskip
\bigskip
\bigskip

Synthèse : Déterminons maintenant les fonctions $y$ solutions. Soit $y\in\Ss$. On a $y$ dérivable et :

\begin{align*}\begin{split}
\forall x\in\R, y'(x)=\frac{-3\lambda x^2+2x}{(\lambda x^3+x^2)^2}
\end{split}\end{align*}

\begin{align*}\begin{split}
xy'+3y&=x^2y^2\\\\
\Longleftrightarrow \frac{-3\lambda x^3 + 2x^2}{(\lambda x^3+x^2)^2}+\frac{3}{\lambda x^3+x^2}&=\frac{x^2}{\lambda x^3+x^2}\\\\
\Longleftrightarrow \frac{-3\lambda x^3 + 2x^2}{\lambda x^3+x^2}+3&=x^2\\\\
\Longleftrightarrow -3\lambda x^3 + 2x^2 +3\lambda x^3 + 3\lambda x^2 - x^2&=0\\\\
\Longleftrightarrow 3\lambda x^2+x^2 &= 0\\\\
\Longleftrightarrow 3\lambda + 1 &= 0\\\\
\Longleftrightarrow& \boxed{\lambda = -\frac{1}{3}}
\end{split}\end{align*}

Finalement, l'ensemble de l'équation $xy'+3y=x^2y^2$ est :

\begin{align*}\begin{split}
\Ss=\Bigl\{\frac{1}{-\frac{1}{3}x^3+x^2}\Bigl\}\\\\
\Longrightarrow \boxed{\Ss=\Bigl\{-\frac{3}{x^3+3x^2}\Bigl\}}
\end{split}\end{align*}


\end{document}