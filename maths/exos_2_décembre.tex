\documentclass{article}

\input{lib.tex}

\title{Exercices Du TD de Mathématiques}
\author{Arnaud, Hugo, Ki Mi, Raphaël, Yoan}

\begin{document}
\maketitle

\underline{Exercice 9 :}

1) Soit $a$ un élément de $G$ d'ordre fini. On pose alors :

\begin{align*}\begin{split}
\exists n\in\N^*/a^n=e \Longrightarrow (a^{-1})^n=(a^n)^{-1}=e
\end{split}\end{align*}

Ainsi, $a^{-1}$ est d'ordre fini. Notons $p\in\N^*$ l'ordre de $a^{-1}$, tel que $\boxed{n\leq p}$. On a alors puisque $a^{-1}$ est d'ordre $p$ :

\begin{align*}\begin{split}
(a^{-1})^p=e \Longrightarrow a^p=((a^{-1})^{-1})^p=((a^{-1})^p)^{-1}=e^{-1}=e
\end{split}\end{align*}

Ainsi, $a$ est d'ordre inférieur ou égal à $p$, donc $\boxed{n\leq p}$.

Finalement, on a : $\boxed{n=p}$.

\bigskip

2) Soit $(a, b)\in G^2$, tels que $a$ est d'ordre fini. On pose :

\begin{align*}\begin{split}
\exists n\in\N^*/a^n=e \Longrightarrow (bab^{-1})^n = ba^nb^{-1}=beb^{-1}=bb^{-1}=e 
\end{split}\end{align*}

Ainsi, $bab^{-1}$ est d'ordre fini, que l'on notera $p$, avec $\boxed{p\leq n}$.
On pose ensuite :

\end{document}