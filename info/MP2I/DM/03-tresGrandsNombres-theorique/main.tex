\documentclass{article}

\input{lib.tex}

\title{DM d'informatique : Partie théorique}
\author{Yoan de CORNULIER}
\date{}

\begin{document}
\maketitle

\bigskip
\bigskip

\underline{I) Définition}

\bigskip
\bigskip

0) Il faut $\log(10^{10^100})=10^{100}$ chiffres pour écrire un gogolplex.

\bigskip
\bigskip

1) Il faut $\log_2(2^{2^p})=2^p$ pour écrire $x(p)$. Ainsi, on a un débordement lorsque :

$$2^p\geq63\Longrightarrow p\geq\log_2(63)\approx6$$

\bigskip
\bigskip

2) La fonction $x$ réalise une bijection de $\R^+$ dans $\R^+$, donc $\forall n\geq2,\exists p\in\N, 2^{2^p}\leq n\leq 2^{2^{p+1}}$. On pose la division euclidienne de $n$ par $x(p)$ :

$$
\exists! (d, g)\in[|1; n|]\times[|0; x(p)[|, n = x(p)d + g
$$

Or $d< x(p)$, on peut donc en conclure que la proposition est vraie $\forall n\geq2$.

\bigskip
\bigskip

3) On remarque que dans le pire des cas, l'arbre est limité par la décroissance du paramètre $g$ de chaque noeud, qui est borné par $x(p)$. Ainsi, une borne supérieure asymptotique est $ll(n)=\log_2(\log_2(n))$.
% Démontrer que la borne est asymptotiquement atteinte.

\bigskip
\bigskip

4) Cette limitation permet d'éviter le débordement de la valeur $2^k$ dans la définition de $(v)$, en effet, au delà de 61, on a :

$$
u_n \mod 2^{62} < 2^{62}\\
\Longrightarrow v_{62,n}< 2^{62} + 2^{62} = 2^{63}
$$

\bigskip
\bigskip

5) Par définition, on a :

$$
h(n+1)=h(n)+x(h(n))\times h(n)
$$

Par tests successifs, $\forall n\geq3$, h(n) déborde. En effet, $h(2)=21474836485$, et alors h(3) calcule $2^{2^{21474836485}}$ qui déborde.

\bigskip
\bigskip

6) Montrons par récurrence que $h(n)$ est impaire $\forall n\in\N$.

\begin{itemize}
    \item Pour $n=0$, $h(0)=1$ est pair.
    
    \item Soit $n\in\N$ tel que $h(n)$ est impair, alors $x(h(n))$ est pair, donc $x(h(n))h(n)$ est pair, donc $h(n+1)$ est impair.
    
    \item En somme, $\forall n\in\N, h(n)$ est impair.
\end{itemize}

\bigskip
\bigskip

15) Un algorithme possible est :

\begin{lstlisting}[language=Python]
tree increment(tree Node(g, p, d)) {
    si g < x(p) {
        g+=1
    } sinon {
        d+=1
        si trouve_p d > 0 {
            p+=1
            d = 1
        }
    }
}
\end{lstlisting}

\end{document}